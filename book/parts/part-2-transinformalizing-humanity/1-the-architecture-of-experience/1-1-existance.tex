We must begin with \definition{existance}{We say that an object $q$ \textbf{exists}, we are generally referring to its membership in the universe $q \in \mathcal{U}$\footnote{The universe is the evolving class of all classes. See \cite{langan CTMU paper} for why $\mathcal{U} \in \mathcal{U}$}} for no \definition{phenomena}{A phenomenon is an observable property $o \in \mathcal{O}$} can occur without it. Indeed, such is tautological, for the very contrast we identify by a phenomenon (as opposed to a mere \definition{noumenon}{A noumenon is an unobservable property"\cite{Webster's dictionary entry} $s \in \mathcal{S}$}) is that between the observed and the hidden.\cite{Kant's critique of pure reason distinguishing phenomenon from noumenon}.

Similarly, with our existance forces us to draw a boundary between that which we are $q_{self} \in \mathcal{U}$ and that which we are not $\mathcal{Q}_{other} = \mathcal{U} \\ q_{self}$, the Markov blanket $p(s\' \bar o, s)$, $p(o\' \bar o, s)$ between our model $\mu_{self}(o\' \bar o)$ and its nature $\mu_{env}(s\' \bar s)$, and considering the other side, one may say 

From the Universe $\mathcal{U}$'s perspective, our own existance $a_{self} \in \mathcal{U}$ identifies a subset of others that we are not, and from our perspective, our existance $ \mathcal{A}_self$ identifies a 
As we will be switching reference frames frequently, we introduce the follow notation to maintain clarity on which perspective is being taken: \[ TODO make a bidirectional \iff relation between the subjective perspective and the objective perspective \refframe{  }{\mathcal{U}}\] with an outer \[ \refframe{ \cdot }{\mathcal{U}}\] implicit if unspecified.





% Likewise by definition of our own \definition{existance}{We say that an object $q$ \textbf{exists}, we are generally referring to its membership in the universe $q \in \mathcal{U}$\footnote{The universe is the evolving class of all classes. See \cite{langan CTMU paper} for why $\mathcal{U} \in \mathcal{U}$}} we identify 




% For clarity, 


% The Markov blanket especially reveals

% Permenance

% Existance, by its very nature, identifies a boundary layer of non-existance. This boundary layer is not a physical boundary, but a conceptual one. It is the line that separates what is, from what is not. This boundary is not static, but dynamic, constantly shifting and changing as our understanding and perception of the world evolves. It is this boundary that gives meaning to the concept of existance, for without it, there would be no contrast, no differentiation between existance and non-existance.



% Related is the establishment of the boundary layer existance. The 


% For a thing to exist i

% Existance establishes a boundary
