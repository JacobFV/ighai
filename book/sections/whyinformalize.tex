If only we had a gold model of the universe! Such a model would provide the ultimate blueprint for artificial general intelligence (AGI), capturing every nuance, every law, and every interaction governing our existence. The tantalizing regularities of the universe (of which intelligence is useless without) suggest that, perhaps somewhere, such a model might exist, holding keys to countless mysteries. However, attempts to access and weild this model have thus far forked along two roads: the formal and the informal.

The former, a theory-heavy rigorous approach, seeks to quantify and codify every phenomenon into precise mathematical formulations. It is the realm of physicists, mathematicians, and scholars, where the language of equations and logic is used to describe the world around us. Grand theories such as general relativity, quantum mechanics, and string theory are all products of this formal approach. These models, while highly effective in their domains, often come with limits to their applicability or require complex mathematical apparatus to be fully grasped.

The informal approach, on the other hand, relies on intuition, heuristics, and qualitative understanding. It is the domain of philosophers, artists, and storytellers. Through narratives, metaphors, and analogies, they capture the essence of the universe and the human experience in it. While they may lack the precision of formal models, they often offer a broader, more holistic view of existence. They appeal to our senses and emotions, making the vast complexities of the universe more relatable and comprehensible.

These two approaches, while seemingly at odds, are in fact complementary. The formal provides the structure and precision necessary for prediction and technology advancement, while the informal gives meaning, context, and a sense of wonder to our existence. Together, they offer a more complete understanding of the universe.

However, neither approach has yet yielded a complete "gold model" of the universe. Both have limitations. The formal, with its reliance on mathematical rigor, often struggles with phenomena that are inherently chaotic or nonlinear. It also faces challenges when trying to unify disparate theories into a single cohesive framework. The informal, while powerful in conveying meaning, can sometimes be too vague or subjective to offer concrete solutions to specific problems. Sadly, as we delve deeper into the intricacies of existence, we inevitably find both — be they equations or narratives — failing to explain or predict our observations. The universe, in its infinite complexity, presents challenges not surmounted by one alone.

And yet, such a bridge is required anytime our conscious minds interact with the world around us. 


Despite these challenges, curiosity perseveres. With advancements in technology, especially recent advances in AI towards the totipotent ideal of artificial general intelligence (defined for our purposes as a system with infinite generalization capacity \ref{def:agi}), there is hope that we might one day be able to bridge the gap between the formal and informal, merging the precision of equations with the richness of narratives. Such a model would not only unlock the secrets of the universe but also reshape our understanding of and interaction with reality itself.

It is precisely this self-





The first roads

the tangible, unpredictable world of reality and the structured, analytical domain of formality.

At a glance, these realms appear diametrically opposed. Reality exists, not because we will it so, but because it inherently does. By contrast, the formal world is a product of intention, meticulously crafted, defined, and willed into existence. 

In both cases, there exists the potential to unfold intricacy beyond our imagination, eg, when either realm violates our expectation of its behavior, but such is _____ in the case of reality, whereas insolvent formalisms are often simply discarded or proven impossible.


The real world, though showing order, is seemingly plauged by spontaneity, chaos, and fluidity. Too, it exists, not because we will it so, but because it inherently does. In contrast, the formal world is a product of intention, meticulously crafted, defined, and willed into existence. Every algorithm, structure, and definition in this domain is a testament to our desire to make sense of the world around us. Yet, despite its precision, formality is often an approximation, a shadow of the rich tapestry of reality.

This dichotomy brings forth a plethora of challenges. While we strive to capture the vibrant hues of reality within the black and white sketches of formality, we invariably face losses. Not everything that exists in the real world can be neatly packaged into our formal constructs. Recognizing these limitations becomes paramount. We must not only acknowledge what our formalisms can capture but also remain acutely aware of what they omit.

One of the inherent truths we grapple with is the nature of existence within these domains. Informalisms, or the entities of the real world, exist simply because they do; they are self-evident and self-sustaining. On the other hand, formalisms are birthed from our conscious efforts, willed into existence to serve as tools to navigate, measure, and predict the real world. They are our best attempts to echo the melodies of the universe in a language we understand.

Artificial Intelligence, as a field, finds itself at the crossroads of this duality. While AI systems strive to mirror and interact with the multifaceted reality, they remain bound by the chains of formalization. Every algorithm, every model is a manifestation of our willingness to formalize, but it's essential to remember that these are mere approximations. We can optimize only within the confines of our formal structures, and as we push these boundaries, we must constantly recalibrate our understanding of the real world.

Amidst this backdrop, we introduce "Transinformalization," a paradigmatic shift aiming to bridge the vast expanse between the organic intricacies of reality and the deliberate constructs of formality. This paper seeks to explore this novel approach, hoping to usher in a new era where we can more effectively navigate the dance between what is and what we perceive.


And please escuse the conflation, but we must point out this motif across the sciences:

\begin{itemize}
    \item Physics: System vs. Model
    \item Biology: Evolution vs. Intelligent Design
    \item Psychology: Nature vs. Nurture
    \item Philosophy: Idealism vs. Realism
    \item Statistics: Frequentist vs. Bayesian
    \item Mathematics: Intuition vs. Rigor
    \item Computer Science: Formalism vs. Informalism
    \item AI: Symbolic vs. Subsymbolic
\end{itemize}