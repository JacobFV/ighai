- noise
- signal
- information
- insight (action informative)
- thought
- personality

- zipf's law: word-frequency 80-20 rule

---

- how can you tell the difference between things?
    - compare two things and then consider the difference

s_the: instantiation
s_I: the (self)
s_a: instantiation (indeterminate)
s_it/that: instantiation (determinate)
s_be: assignment
s_of: flowing from X
s_to: flowing to X
s_and: symbols existing simultaneously
s_in: membership
s_have: possession (membership from the parent)
S_core = \{s_the, s_I, s_a, s_it, s_be, s_of, s_to, s_and, s_in, s_have\}
S_other = \{\}
S = S_core \cup S_other

- Artificial syndifeonesis: the ability to adjust a partition function to accommodate a new symbol


s_obj === f_emb(G_a)

```python

class Modality:
    pass

class InputModality(Modality):
    def get(self) -> T:
        pass

class OutputModality(Modality):
    def set(self, value: T):
        pass

class InputOutputModality(InputModality, OutputModality):
    pass


l_eye, r_eye, l_ear, r_ear, mouth, skeletal_movement = ...

SIMILARITY_EARLY_EXIT_THRESHOLD = 0.8

class Symbol:
    children: Symbols[]

    @abstractmethod
    def abstraction_gen(self) -> Generator[Symbol, None, None]:
        pass

    def compare(self, other) -> Symbol:
        self_abs_gen = self.abstraction_gen()
        other_abs_gen = other.abstraction_gen()

        for self_abs, other_abs in zip(self_abs_gen, other_abs_gen):
            if self_abs == other_abs:
                return self_abs
        
        delta = self_abs - other_abs
        return AtomicSymbol(delta)

class AtomicSymbol(Symbol):
    emb: Tensor

    def abstraction_gen(self) -> Generator[Symbol, None, None]:
        yield self

class Graph(Symbol):
    verts: Symbol[]
    edges: list[Tuple[Symbol, Symbol]]

    # @cached_property
    # def emb(self) -> Tensor:
    #     # some function of the graph
    #     ...

    def subgraphs(self) -> Generator(Graph):
        ...

    def sim(self, other) -> float:
        maxsim = 0

        if self == other:
            maxsim = 1.0

        for children in self.children:
            maxsim = max(maxsim, children.sim(other))
            if maxsim > SIMILARITY_EARLY_EXIT_THRESHOLD:
                return maxsim

        for subgraph in self.subgraphs():
            subgraph_percentage = (len(subgraph.nodees) / len(self.nodes)) * (len(subgraph.edges) / len(self.edges)
            maxsim = max(maxsim, subgraph_percentage*subgraph.sim(other))
            if maxsim > SIMILARITY_EARLY_EXIT_THRESHOLD:
                return maxsim

        return maxsim

    def abstraction_gen(self) -> Generator[Symbol, None, None]:
        yield self

        while len(edges) > 0:
            # weight the edges based on the degree of the node
            # stochastically remove the lowest-weighted edge
            # yield the resulting graph
            ...

"""
f_A: obs , wmg -> pred, wmg'
f_B: obs , wmg -> pred, wmg'

f_A(f_B(f_A(...)))


# return the number of similarity of subgraphs in the graph
# the more isomorphic causal subgraphs, the more powerful it is
# consider a subset of a graph as being a particular word or idea
# you want to increase to isomorphism between internal representations and external concepts

nA -> nB: cause to effect

innacurate comparisons is when a subset matches but its neighboring context isn't relevant

a subgraph that's able to fit into a graph that being built
- MTM: the scope of symbol under unconscious consideration for subgraph matching
- STM:
    - symbols available to condition downstream functions
    - determine the subgraph isomorphism comparison precedence
"""

# metaphor finds similar essence in different concepts

class Brain:

    modalities: Modality[] = l_eye, r_eye, l_ear, r_ear, mouth, skeletal_movement

    s_ltm: GraphSymbol # all known information, requires procedural access for scale efficiency reasons
    s_mtm: GraphSymbol # symbols accessible for matching, but not for testing all permutations over
    s_stm: GraphSymbol # continued ripple of perception, declarative/implicit matching baked into the perception process

    def perception(self):
        encs = [modality.encode(modality.get()) for modality in self.modalities]
        s_obs = AtomicSymbol(self.fuse(encs))

        # if s_obs doesn't exist in s_mtm, create it:
        for vert in self.s_mtm.verts:
            if vert.sim(s_obs) > THRESHOLD:
                return
        self.s_mtm.verts.append(s_obs)

        # use the symbols in the stm to compose a symbol for the s_obs that matches simething in mtm
        for s_subgraph in self.s_stm.subgraphs:
            if s_subgraph.sim(s_obs) > THRESHOLD:
                s_mtm.connect(s_obs, s_subgraph)
                return

1 ---> 2 ---> 3 ---> 4

Y ---> Z


# TODO: "thing" is the root node of all nodes
# TODO: the "identity implication" applies to the root node, and it is the fundamental operator

# the directed edge is the only operator

# the implications of a chain of parent nodes would differ from another chain

# the combination of nodes defines what the operator is doing
# 1+1=2 === 1->G_add_1=2
# 1 -> G_add -> G_add_1 = G_inc
# 5 -> G_inc -> 6
# loop through a subgraph x number of times
```

```text
G = <V, E>

- go to the last node in the subgraph, if it wants to iterate it will point to its 


```


s1 -> s2
s2 -> s3
g1 <- s1
g1 <- s2
g1 <- s3
g1 -> s_obs


-------------

```
the final node of the increment subgraph would be the copy of the increment subgraph
represent the values as nodes themselves

g_digits0-9 = {0, 1, 2, 3, 4, 5, 6, 7, 8, 9}

g_digits_bin = {0, 1}

3 g-digit bins and then assign a binary literal to each bin

1     0     1
v1 -> v2 -> v3
v1 -> v1
v3 -> v3

if it recieves the function 1+1, it will cal the appropriate code function

however many nodes in the subgraph is the same os how many iterations it will do

clearly define how to represent a number in graph format

1     1     0
v1 -> v2 -> v3
v1 -> v1
v2 -> v2



a1 -> a2 -> a3 -> a4
b1 -> b2 -> b3 -> b4
X , Y -> opp_add -> Z
A + B + C = (A+B)+C

a4 , b4 -> opp_add

(a1 -> a2 -> a4 , b1 -> b2 -> b3 -> b4 -> opp_add) -> 
    (if a4 == 0 and b4 == 0 then modify nothing an copy into c4, repeat on a3, b3)
    (if a4 == 0 and b4 == 1 then modify a2)
    ...
    ...




1. construct a new subgraph matching 1+1 and then make that subgraph match to 2

exponential notation would enable use to have repetitive strings for the primray nubmers

2: 11
3: 111

G0 = {V0, E0}
V0 = {} + the g0 root node
E0 = {}

G1 = {V1, E1}
V1 = {g0, v1} + the g1 root node
E1 = {(g0, g1), (v1, g1)}

G2 = {V2, E2}
V2 = {g1, v2} + the g2 root node
E2 = {(g1, g2), (v2, g2)}

G3 = {V3, E3}
V3 = {g2, v3} + the g3 root node
E3 = {(g2, g3), (v3, g3)}








```